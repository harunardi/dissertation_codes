% For copyright and license information, see uiucthesis2021.dtx and derivatives.
\documentclass{uiucthesis2021}
\usepackage[utf8]{inputenc}
\usepackage[english]{babel}
\usepackage{csquotes}
\usepackage{microtype}
\usepackage{amsmath,amsthm,amssymb}
\usepackage[bookmarksdepth=3,linktoc=all,colorlinks=true,urlcolor=blue,linkcolor=blue,citecolor=blue]{hyperref}
\usepackage[capitalize]{cleveref}
\usepackage[style=ieee]{biblatex}

% \usepackage{ruledchapters}  % example of compliant heading format, uncomment to use

\usepackage{lipsum}  % just for placeholder code

% uncomment the below to show a grid on all pages
% \usepackage[grid, gridunit=in, gridcolor=blue!40, subgridcolor=blue!20]{eso-pic}

\addbibresource{./references.bib}

\newcounter{counterforappendices}

\begin{document}

\title{A Reactor Physics Framework to Detect Anomalies in HTGR Core}
\author{Harun Ardiansyah}
\department{Nuclear, Plasma, and Radiological Engineering}
%\concentration{Coffee Studies}
\phdthesis
\degreeyear{2025}
\committee{
    Professor Tomasz Kozlowski, Director of Research, Chair\\
    Professor Angela Di Fulvio\\
    Assistant Syed Bahauddin Alam\\
    Professor Eduard-Wilhelm Kirr\\
    Professor Christophe Demaziere}
\maketitle

\frontmatter

\begin{abstract}
This is a comprehensive study of caffeine consumption by graduate
students at the University of Illinois who are in the very final
stages of completing their doctoral degrees. A study group of six
hundred doctoral students\ldots.
\end{abstract}

\begin{dedication}
To my parents, for their unwavering support and encouragement. To Nafi, the love of my life. To Ali, who makes my life complete.
\end{dedication}

\begin{acknowledgments}
This dissertation is ...This project would not have been possible without the support of
many people. Many thanks to my adviser, Lawrence T. Strongarm, who
read my numerous revisions and helped make some sense of the
confusion. Also thanks to my committee members, Reginald Bottoms,
Karin Vegas, and Cindy Willy, who offered guidance and support.
Thanks to the University of Illinois Graduate College for awarding
me a Dissertation Completion Fellowship, providing me with the
financial means to complete this project. And finally, thanks to
my husband, parents, and numerous friends who endured this long
process with me, always offering support and love.
\end{acknowledgments}

{
    \hypersetup{linkcolor=black}  % disable link coloring locally
    \tableofcontents
    % the Graduate College doesn't recommend including lot or lof
    % \listoftables
    % \listoffigures
}

\chapter{List of Abbreviations}

\begin{abbrevlist}
\item[CA] Caffeine Addict.
\item[CD] Coffee Drinker.
\end{abbrevlist}

\chapter{List of Symbols}

\begin{symbollist}[0.7in]
\item[$\tau$] Time taken to drink one cup of coffee.
\item[$\mu$g] Micrograms (of caffeine, generally).
\end{symbollist}

\mainmatter

\chapter{Introduction}

\section{Motivation}

Core diagnostics have been a key component of nuclear reactor operation to ensure reactor safety and performance. As a multi-component system, perturbations in nuclear reactors are unavoidable. In the past, neutron noise experiments were developed to study the effect of perturbations in zero power reactors \cite{akcasuApplicationLangevinTechnique1966,cohnSimplifiedTheoryPile1960,moorePowerNoiseTransfer1959}. Neutron noise is defined as the stochastic or random process that always happens in a nuclear reactor [6]. Another meaning also defines noise as the fluctuation in the output of the detector when the incident radiation is steady [7]. The results show that the neutron noise method could be used to detect anomalies in zero power reactors. Later, the neutron noise method was used to detect anomalies in power reactors. Some examples include continuous neutron noise monitoring at the High Flux Isotope Reactor (HFIR), neutron- and pressure noise monitors at the Molten Salt Reactor Experiment (MSRE) [8], neutron noise diagnostics in the Palisades Nuclear Generating Station in Michigan [9], and German measurements in boiling water reactors (BWR) that showed vibrations quantified using neutron noise simulations [10]. 

The success of core diagnostics using neutron noise experiments motivates the development of computational models of neutron noise. The computational model of neutron noise was at first developed for zero power reactors, similar to the experiments. The early computational model introduces the concept of noise equivalent source [2]. Then, [1] detailed further how this concept of noise equivalent source can be included in the neutron transport equation. In the paper, [1] developed the space- and energy-dependent theory of the neutron noise using the Langevin technique. The Langevin technique has been extensively used in studies such as Brownian motion and thermodynamics. [1] also evaluated the noise equivalent source as fluctuations of all the phenomena that contribute to the neutron transport equation. That includes capture, fission, scattering, and external sources. To validate the application of Langevin technique, the paper included the correlations of the count rates of two detectors at two different phase points in a zero-power reactor system. Further investigation by [11] took advantage of the Langevin technique to model the two group neutron noise analysis. The paper suggested the simplification of neutron transport into two group neutron diffusion theory and applied the Langevin technique to the said theory. In the two-group application, [11] reported that the space independent model and explicit expressions for the auto- and cross-power spectral densities of the two groups can be obtained. These results show a consistent correlation between the noise source and the power spectral density in the measurement of neutrons from experiments of a zero-power reactor system. 

Further development of the model led to the concept of noise unfolding method, which is a method to detect the location of neutron noise and determine the magnitude of neutron noise. There are three main methods that have been developed to unfold neutron noise: inversion method, zoning method, and scanning method. All of the methods require the Green’s function matrix to solve the problem [12]. The input for these methods is the noise fluxes from the detector readings, and the outputs are the noise locations and magnitude.
In this work, computational models of neutron noise are developed based on neutron diffusion equation in frequency domain. Solver is developed using box-scheme finite difference for rectangular and hexagonal geometries. The main goal of the solver is application to HTGRs. However, application to rectangular geometries is also provided to differentiate the characteristics of HTGRs and LWRs systems. Code-to-code comparisons are provided in this work. Methods to unfold neutron noise are also developed in the simulator to highlight the advantages and disadvantages of the methods. 


\section{Objectives}

\section{Outline}

How does this relate to coffee? We direct the reader to Refs.~\cite{Trembly98,Childish07,Presso10}.

Equations are numbered within chapters:
\begin{align}
    a
    &= b + c \\
    d
    &= \frac{e}{f}.
\end{align}

% input individual chapters files
% \input{1-introduction}
% \input{2-related}
% \input{3-model}
% \input{4-predictions}

\chapter{Literature Review}

\chapter{Simulation of Neutron Noise in 2D and 3D HTTR}

\chapter{Application Neutron Noise Unfolding in 2D and 3D HTTR}


\chapter{Novel Neutron Noise Unfolding methods and applications}


\chapter{Conclusions}

We conclude that graduate students like coffee.

% per Graduate College preference, place the \appendix and the appendices content before the
% bibliography (here) only if the appendices contain references.

\backmatter

\printbibliography[heading=bibintoc,title={References}]

% the below lines are only needed if bibliography precedes appendices
% uses https://tex.stackexchange.com/a/440212 to continue page numbering
\clearpage
\setcounter{counterforappendices}{\value{page}}
\mainmatter
\setcounter{page}{\value{counterforappendices}}

\appendix

\chapter{An appendix}

\lipsum[1-5]

% \input{Appendix.tex}

\end{document}
