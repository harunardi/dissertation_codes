\label{ch:intro}

\section{Motivation}

The development of advanced reactors has been rising for the last several years. This is backed with the support of the financial market through startups and private funds, and also supportive public perception of nuclear in recent years. With the 

Core diagnostics have been a key component of nuclear reactor operation to ensure reactor safety and performance. As a multicomponent system, perturbations in nuclear reactors are unavoidable. In the past, neutron noise experiments were developed to study the effect of perturbations in zero-power reactors \cite{akcasuApplicationLangevinTechnique1966,cohnSimplifiedTheoryPile1960,moorePowerNoiseTransfer1959}. Neutron noise is defined as the stochastic or random process that always happens in a nuclear reactor \cite{saitoTheoryPowerReactor1974b}. Another meaning also defines noise as the fluctuation in the output of the detector when the incident radiation is steady \cite{pacilioAnalysisReactorNoise1979}. The results show that the neutron noise method could be used to detect anomalies in zero-power reactors. Later, the neutron noise method was used to detect anomalies in power reactors. Some examples include continuous neutron noise monitoring at the \gls*{HFIR}, neutron and pressure noise monitors at the \gls*{MSRE} \cite{fryExperienceReactorMalfunction1971}, neutron noise diagnostics at the Palisades Nuclear Generating Station in Michigan \cite{fryAnalysisNeutrondensityOscillations1975}, and German measurements in \gls*{BWR} that showed vibrations quantified using neutron noise simulations \cite{wachInvestigationJointEffect1974}. 

The success of core diagnostics using neutron noise experiments motivates the development of computational models of neutron noise. The computational model of neutron noise was developed first for zero-power reactors, similar to the experiments. The early computational model introduces the concept of noise equivalent source \cite{cohnSimplifiedTheoryPile1960}. Then, \cite{akcasuApplicationLangevinTechnique1966} further detailed how this concept of a noise equivalent source can be included in the neutron transport equation. In the paper, \cite{akcasuApplicationLangevinTechnique1966} developed the space and energy-dependent theory of neutron noise using the Langevin technique. The Langevin technique has been extensively used in studies such as Brownian motion and thermodynamics. \cite{akcasuApplicationLangevinTechnique1966} also evaluated the equivalent source of noise as fluctuations of all the phenomena that contribute to the neutron transport equation. That includes capture, fission, scattering, and external sources. To validate the application of Langevin technique, the paper included the correlations of the count rates of two detectors at two different phase points in a zero-power reactor system. Further investigation in \cite{nietoTwogroupReactorNoise1968} took advantage of the Langevin technique to model the analysis of two groups of neutron noise. The paper suggested the simplification of neutron transport into two-group neutron diffusion theory and applied the Langevin technique to the said theory. In the application of two groups, \cite{nietoTwogroupReactorNoise1968} reported that the space-independent model and explicit expressions for the auto- and cross-power spectral densities of the two groups can be obtained. These results show a consistent correlation between the noise source and the power spectral density in the measurement of neutrons from experiments of a zero-power reactor system. 

Further development of the model led to the concept of noise unfolding method, which is a method to detect the location of neutron noise and determine the magnitude of neutron noise. There are three main methods that have been developed to unfold neutron noise: the inversion method, the zoning method, and the scanning method. All of the methods require the Green function matrix to solve the problem \cite{pazsitNoiseTechniquesNuclear2010}. The input for these methods is the noise fluxes from the detector readings, and the output is the noise locations and magnitude.

In this work, computational models of neutron noise are developed based on the neutron diffusion equation in the frequency domain. The solver is developed using the box-scheme finite difference for rectangular and hexagonal geometries. The main goal of the solver is application to \glspl*{HTGR}. However, application to rectangular geometries is also provided to differentiate the characteristics of \glspl*{HTGR} and \glspl*{LWR} systems. In this work, code-to-code comparisons are provided. Methods for unfolding neutron noise are also developed in the simulator to highlight the advantages and disadvantages of the methods. 


\section{Objectives}

The main objective of this work is to develop and demonstrate core diagnostics in HTGR system using neutron noise method. The following objectives are identified to advance this goal:
\begin{enumerate}
    \item Develop numerical solver to perform forward, adjoint, and noise (in frequency domain) calculation.
    \item Determine the characteristics of noise in HTGRs compared to LWRs through the zero-power reactor transfer function.
    \item Determine the distinct characteristics of noise sources in LWRs and HTGRs, namely absorber of variable strength and fuel assembly vibration.
    \item Perform noise unfolding in prismatic HTGRs using existing methods.
    \item Explore improvements in noise unfolding methods through noise flux interpolation
\end{enumerate}

Objective (1) focuses on the development of a numerical solver that could perform simulations in rectangular and hexagonal geometries using provided macroscopic cross-sections. The numerical solver is necessary for the overall goal, especially for hexagonal geometries using box-scheme finite difference. This solver will be used to model and simulate neutron noise in HTGR systems, which is the focus of objective (2). The HTGR model used in this case is the HTTR benchmark model. The HTTR benchmark is modeled in Monte Carlo code Serpent \cite{leppanenSerpentMonteCarlo2015}. This Serpent model is a heterogeneous model with various materials. The details of the HTTR model will be provided in the following sections. This model is mainly used to generate homogenize cross-sections for the solver. 

After obtaining the model, objective (3) and (4) focuses on modeling various neutron noise models and unfolding methods that existed in literature. These models and methods are mainly developed for LWR systems. However, it is meaningful to apply the methods and model to understand the different behavior of HTGR system under small perturbations. The novelty of this works relies on objective (5). Using all the information known from previous objectives, it is important to extend these models and methods to reflect the important physics of HTGR system and leveraging all the known parameters from forward and adjoint calculations.

\section{Outline}

How does this relate to coffee? We direct the reader to Refs.

Equations are numbered within chapters:
\begin{align}
    a
    &= b + c \\
    d
    &= \frac{e}{f}.
\end{align}

% input individual chapters files
% \input{1-introduction}
% \input{2-related}
% \input{3-model}
% \input{4-predictions}
