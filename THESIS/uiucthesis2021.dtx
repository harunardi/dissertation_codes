% \iffalse meta-comment
%
% Copyright 2021 Zachary J. Weiner
% Version 3.x was derived (with substantial revision) from a prior version by
% Stephen Mayhew available at https://github.com/mayhewsw/uiucthesis2014,
% which indicates prior authorship by Charles Kiyanda, Tim Head, Peter Czoschke,
% and David Hull.
% No copyright statement or license was included in versions prior to 3.1.
%
% This work may be distributed and/or modified under the
% conditions of the LaTeX Project Public License, either version 1.3
% of this license or (at your option) any later version.
% The latest version of this license is in
%   http://www.latex-project.org/lppl.txt
% and version 1.3 or later is part of all distributions of LaTeX
% version 2005/12/01 or later.
%
% This work has the LPPL maintenance status “maintained”.
%
% The Current Maintainer of this work is Zachary J. Weiner.
%
% This work consists of the files uiucthesis2021.dtx and the derived files
% uiucthesis2021.{sty,ins,cls,pd}, and example files thesis.tex,
% ruledchapters.sty, and references.bib.

% !TEX program  = pdfLaTeX
%<*internal>
\iffalse
\fi
\def\nameofplainTeX{plain}
\ifx\fmtname\nameofplainTeX\else
  \expandafter\begingroup
\fi
%</internal>
%<*install>
\input docstrip.tex
\keepsilent
\askforoverwritefalse
\preamble

Copyright 2021 Zachary J. Weiner
Version 3.x was derived (with substantial revision) from a prior version by
Stephen Mayhew available at https://github.com/mayhewsw/uiucthesis2014,
which indicates prior authorship by Charles Kiyanda, Tim Head, Peter Czoschke,
and David Hull.
No copyright statement or license was included in versions prior to 3.1.

This work may be distributed and/or modified under the
conditions of the LaTeX Project Public License, either version 1.3
of this license or (at your option) any later version.
The latest version of this license is in
  http://www.latex-project.org/lppl.txt
and version 1.3 or later is part of all distributions of LaTeX
version 2005/12/01 or later.

This work has the LPPL maintenance status “maintained”.

The Current Maintainer of this work is Zachary J. Weiner.

This work consists of the files uiucthesis2021.dtx and the derived files
uiucthesis2021.{sty,ins,cls,pd}, and example files thesis.tex,
ruledchapters.sty, and references.bib.

\endpreamble

\usedir{tex/latex/demopkg}
\generate{
  \file{\jobname.sty}{\from{\jobname.dtx}{package}}
  \file{\jobname.cls}{\from{\jobname.dtx}{class}}
}
%</install>
%<install>\endbatchfile
%<*internal>
\usedir{source/latex/demopkg}
\generate{
  \file{\jobname.ins}{\from{\jobname.dtx}{install}}
}
\ifx\fmtname\nameofplainTeX
  \expandafter\endbatchfile
\else
  \expandafter\endgroup
\fi
%</internal>

%<*class|package>
\def\fileversion{v3.1}
\def\filedate{2021/09/01}
\NeedsTeXFormat{LaTeX2e}
%</class|package>
% \def\fileversion{v3.1}
% \def\filedate{2021/09/01}

%<class>\ProvidesClass{uiucthesis2021}[\filedate\space\fileversion\space UIUC Thesis]
%<package>\ProvidesPackage{uiucthesis2021}[\filedate\space\fileversion\space UIUC Thesis]

%<*driver>
\documentclass{ltxdoc}
\usepackage[T1]{fontenc}
\usepackage{lmodern}
\usepackage[numbered]{hypdoc}
\EnableCrossrefs
\CodelineIndex
\RecordChanges
\begin{document}
  \DocInput{\jobname.dtx}
\end{document}
%</driver>
% \fi
%
%\title{^^A
%  The \textsf{uiucthesis2021} class\thanks{^^A
%    This file describes version \fileversion, last revised \filedate.^^A
%  }^^A
%}
%\author{^^A
%  Zachary J. Weiner\thanks{
%        This version was adapted from Stephen Mayhew's edit of the version by Charles Kiyanda
%        which itself was adapted from version 2.25, 2005/03/25 by Peter Czoschke
%        and updated in 2007 by Tim Head.
%        It was significantly simplified for the (apparently less restrictive) 2021 requirements and
%        a number of no-longer-relevant options were removed.
%        As many stylistic features/options (i.e., those not specified by the requirements)
%        were removed as possible, the idea being these are more easily controlled by user-created
%        style files.
%    }
%}^^A\thanks{E-mail: you@your.domain}^^A
%\date{Released \filedate}
%
%\maketitle
%
%^^A \changes{v1.0}{2009/10/06}{First public release}
%
% \CheckSum{0}
%
% \MakeShortVerb{\|}
%
% \def\pkg#1{\textsf{#1}}
% \def\env#1{\textsf{#1}}
%
% \begin{abstract}
% A \LaTeX2e documentclass to produce a document intended to conform
% to the format described in the UIUC Graduate College thesis requirements.
% \end{abstract}
%
%
% \section{Using the class}
%
% Load the \pkg{uiucthesis2021} class at the beginning of
% your \LaTeX\ document with the |\documentclass| command.
% The included template \pkg{thesis.tex} file illustrates usage
% (including proper ordering of elements, etc.).
%
% \subsection{The Title Page}
%
% The |\maketitle| command is redefined so that it creates a
% title page with the correct format for a thesis at UIUC.
%
% \DescribeMacro{\phdthesis}
% \DescribeMacro{\msthesis}
% Use the |\phdthesis| or |\msthesis| to set the correct thesis type.
% The default thesis type is |\phdthesis|.
%
% \DescribeMacro{\department}
% Set your department in which your degree will be with\linebreak
% |\department{|\meta{department}|}|, leaving out ``Department of.''
%
% \DescribeMacro{\college}
% Define your college with |\college{|\meta{college}|}|.
% The default is college is ``Graduate College'';
% you shouldn't need to change it.
%
% \DescribeMacro{\concentration}
% \DescribeMacro{\minor}
% Use to specify the name of your concentration or minor
% (the appropriate lines will only appear if defined).
%
% \DescribeMacro{\degreeyear}
% Use |\degreeyear{|\meta{year}|}| to define the year in which
% you will receive your degree. The default is the current year.
%
% \DescribeMacro{\committee}
% Use |\committee{|\meta{committee members}|}| to specify the members
% of your committee and their titles as you want them to appear on the
% title page. Separate members with |\\|. To respect the graduate college guidelines,
% you must use the full title of each committee members. The committee chair should
% appear first with the designation ``, Chair''. Your thesis adviser should appear second
% with the title ``, Director of Research''. See the graduate college website for details.
%
% \subsection{Front Matter}
%
% \DescribeMacro{\frontmatter}
% Typically, a thesis might have an abstract, a dedication, and
% acknowledgments before the table of contents.
% Use the |\frontmatter| command to start this preliminary section
% of the thesis, which sets the page number of the next page
% to roman numeral ii.
%
% \DescribeEnv{abstract}
% The abstract should appear in the \env{abstract} environment.
% This just produces another chapter with |\chapter*{\abstractname}|,
% where |\abstractname| is ``Abstract'' (see User Customization below).
%
% \DescribeEnv{dedication}
% A dedication page can be printed with the \env{dedication} environment.
% This produces a separate page with the dedication centered horizontally
% and vertically, with the text in italics.
%
% \DescribeEnv{acknowledgments}
% Analagous to \env{abstract}.
%
% After this front matter comes the Table of Contents,
% List of Tables, List of Figures, etc.  Use the standard \LaTeX\
% commands |\tableofcontents|, |\listoftables|, |\listoffigures|, etc.,
% to generate them.
% In the \pkg{uiucthesis2021} format these lists are all single spaced.
%
% \DescribeEnv{symbollist}
% \DescribeEnv{abbrevlist}
% Optionally, these tables can be followed by a List of Abbreviations and/or
% List of Symbols. Introduce these with the |\chapter| command. To aid in
% making these lists, the \env{symbollist} and \env{abbrevlist}
% produce a two-column list as illustrated below. By default the left column
% is 1 inch wide but can be specified with an optional argument.
%
% \subsection{Main Matter}
%
% \DescribeMacro{\mainmatter}
% Begin the main body of your thesis with the |\mainmatter| command, which
% resets the page number to arabic numeral 1.
% You can now use any of the commands defined by the
% the book document class to write your thesis.
%
% \subsection{Reference Matter}
%
% \DescribeMacro{\appendix}
% To switch from the body of your thesis to the reference material
% at the end, use the standard \LaTeX\ |\appendix| command.
%
% \subsection{Back Matter}
%
% \DescribeMacro{\backmatter}
% The last few chapters in your thesis should not have chapter
% numbers, but should be listed in the Table of Contents.
% These chapters include the Bibliography and the Index.  \LaTeX's
% |\backmatter| command accomplishes this.
%
%
% \section{User Customization}
%
% The \pkg{uiucthesis2021} class loads \pkg{setspace} for the
% line spacing commands and enables |\onehalfspacing| (the minimum required).
% This default may be overridden within the document body.
%
% The table of contents depth is set to include only chapters (the Graduate
% College's stated preference).
% Override by calling |\setcounter{tocdepth}{\meta{depth}}| again with depth
% argument 1, 2, etc., to include sections, subsections, etc.
%
% No special format is implemented for chapter headings by default.
%
%
% \DescribeMacro{\nocopyrightpage}
% Put the |\nocopyrightpage| macro in the preamble to disable the default
% copyright notice page.
%
%\StopEventually{^^A
%  \PrintChanges
%^^A   \PrintIndex
%}
%
%    \begin{macrocode}
%<*class|package>
%    \end{macrocode}
%
% \section{Implementation}
%
% This section shows the implementation of the \pkg{uiucthesis2021} class.
% Unless you are interested in the details of how \pkg{uiucthesis2021} works,
% you probably don't need to read it.
%
%
% First instruct \pkg{setspace} to use |\onehalfspacing| by default.
% Load the \pkg{book} class with the |[oneside]| and |[letterpaper]| options.
%    \begin{macrocode}
%<*class>
\DeclareOption*{\PassOptionsToClass{\CurrentOption}{book}}
\PassOptionsToClass{letterpaper,oneside}{book}
\ProcessOptions
\LoadClass{book}
%</class>
\usepackage[onehalfspacing]{setspace}
%    \end{macrocode}
%
% \subsection{Title Page}
%
% \begin{macro}{\title}
% \begin{macro}{\author}
% Override the standard definitions of |\title| and |\author| to also
% define uppercased versions.
%    \begin{macrocode}
\def\@mkuptitle#1{\gdef\@Utitle{#1}}
\def\title#1{\gdef\@title{#1}\MakeUppercase{\protect\@mkuptitle{#1}}}
\def\@mkupauthor#1{\gdef\@Uauthor{#1}}
\def\author#1{\gdef\@author{#1}\MakeUppercase{\protect\@mkupauthor{#1}}}
%    \end{macrocode}
% \end{macro}
% \end{macro}
%
% \begin{macro}{\phdthesis}
% \begin{macro}{\msthesis}
% \begin{macro}{\department}
% \begin{macro}{\college}
% \begin{macro}{\degreeyear}
% \begin{macro}{\committee}
% Macros to set title page elements.
%    \begin{macrocode}
\def\phdthesis{\def\@degree{Doctor of Philosophy}
    \def\degree{Ph.D.}
    \def\@thesisname{DISSERTATION}
    \def\@committeename{Doctoral Committee:}
    }
\def\msthesis{\def\@degree{Master of Science}
    \def\degree{M.S.}
    \def\@thesisname{THESIS}
    \def\@committeename{Master's Committee:}
    }
\def\department#1{\def\@dept{#1}}
\def\concentration#1{\def\@concentration{#1}}
\def\minor#1{\def\@minor{#1}}
\def\college#1{\def\@college{#1}}
\def\degreeyear#1{\def\@degreeyear{#1}}
\newcommand{\committee}[1]{\gdef\@committee{#1}}
%    \end{macrocode}
% \end{macro}
% \end{macro}
% \end{macro}
% \end{macro}
% \end{macro}
% \end{macro}
%
% \begin{macro}{\copyrightnotice}
% Define the copyright notice as a macro so that the user
% can change it if desired.
%    \begin{macrocode}
\def\copyrightnotice{\copyright~\@degreeyear~ \@author}
%    \end{macrocode}
% \end{macro}
% \begin{macro}{\nocopyrightpage}
% The printing of the copyright page can also be turned off with the
% |\nocopyrightpage| command (must come before |\maketitle|):
%    \begin{macrocode}
\newif\if@thesiscrpage \@thesiscrpagetrue
\let\nocopyrightpage\@thesiscrpagefalse
%    \end{macrocode}
% \end{macro}
%
% Set the default title page elements.
%    \begin{macrocode}
\phdthesis
\college{Graduate College}
\def\@degreeyear{\number\year}
%    \end{macrocode}
%
% \begin{macro}{\maketitle}
% Redefine \pkg{book}'s |\maketitle| command to produce
% the titlepage in the correct format.
%
%    \begin{macrocode}
\renewcommand\maketitle{
%    \end{macrocode}
% Print the copyright page if we're supposed to.
%    \begin{macrocode}
    \if@thesiscrpage
        \newpage
        \thispagestyle{empty}
        \null\vfill
        \centerline{\copyrightnotice}%
        \vskip 3ex % skip to visually center copyright notice
        \vfill
    \fi
%    \end{macrocode}
% Now start a new page for the title page.  It is single-spaced.
%    \begin{macrocode}
    \newpage
    \thispagestyle{empty}%
    \enlargethispage{1in}%
    \begingroup
    \def\baselinestretch{1}
%    \end{macrocode}
% Check what size font we are using for the text and select a
% smaller size appropriately.
%    \begin{macrocode}
    \ifnum \@ptsize=2
        \@normalsize
        \newcommand{\thesis@small}{\small}
    \else
        \large
        \newcommand{\thesis@small}{\@normalsize}
    \fi
%    \end{macrocode}
% We have to be careful to get the vertical position right.  The
% easiest way to do this seems to be to just set |\topmargin|,
% |\headheight|, and |\headsep| for this page.
%    \begin{macrocode}
    \headheight=0pt \headsep=0pt
    \topmargin=0in
%    \end{macrocode}
% Adjust the horizontal spacing so that the title page
% is centered on the page even if the rest of the document isn't.
% I'm not sure when |\textwidth| changes take place, so instead
% we calculate the correct |\oddsidemargin| to center the text column.
%    \begin{macrocode}
    \@tempdima=\paperwidth
    \advance\@tempdima by -\textwidth
    \divide\@tempdima by 2
    \advance\@tempdima by -1in
    \oddsidemargin=\@tempdima
    \let\evensidemargin=\oddsidemargin

%    \end{macrocode}

% Create the title page as specified by the Grad College.
%    \begin{macrocode}
    \newdimen\thesis@dim
    \thesis@dim=1.5in
    \newdimen\ct@dim
    \newdimen\cn@dim
    \ct@dim=\oddsidemargin
    \advance\ct@dim by -0.3125in
    \cn@dim=\oddsidemargin
    \advance\cn@dim by -0.6875in
    \def\lc@selectfont{}%
    \begin{center}
    \vbox to 1in{}
    \vbox to \thesis@dim{%
        {\lc@selectfont\@Utitle}
        \vfil}%
    \vbox to 0.5in{%
        {\lc@selectfont BY}\\[12pt]
        \vfil}%
    \vbox to 1.5in{%
        {\lc@selectfont\@Uauthor}\\[12pt]
        \vfil}%
    \vbox to 2.0in{%
        {\lc@selectfont \@thesisname}\\[12pt]
        Submitted in partial fulfillment of the requirements\\
        for the degree of \@degree\ in \@dept\\
        \ifdefined\@concentration
            with a concentration in \@concentration\\
        \fi
        \ifdefined\@minor
            with a minor in \@minor\\
        \fi
        in the \@college\ of the\\
        University of Illinois Urbana-Champaign, \@degreeyear\vfil}%
    \vskip -1.5ex%
    \vbox to 0.4in{%
    Urbana, Illinois}%
    \end{center}
	\begin{flushleft}
        \vbox to 0.3in{%
        \hspace{-\ct@dim}\@committeename\\}
        \hspace{-\cn@dim}\begin{tabular}{l}\@committee\end{tabular}\vfil
	\end{flushleft}
    \newpage
    \endgroup
}
%    \end{macrocode}
% \end{macro}
%
% \subsection{Front Matter}
%
% \begin{macro}{\frontmatter}
% Redefine |\frontmatter| so that it sets the page number to 2.
%    \begin{macrocode}
\let\thesis@frontmatter=\frontmatter
\def\frontmatter{%
\thesis@frontmatter
\setcounter{page}{2}}
%    \end{macrocode}
% \end{macro}
%
% \subsection{Table of Contents}
%
% \begin{macro}{\contentsname}
% Title with ``Table of Contents'' instead of ``Contents''.
%    \begin{macrocode}
\RequirePackage[english]{babel}
\addto{\captionsenglish}{\renewcommand*{\contentsname}{Table of contents}}
%    \end{macrocode}
% \end{macro}
%
% \begin{macro}{\tableofcontents}
% We want the Table of Contents to be single-spaced, so
% we save the original definition, and then arrange it so
% that the new |\tableofcontents| calls |\singlespacing| before calling
% the original definition.
%    \begin{macrocode}
\let\thesis@tableofcontents=\tableofcontents
\def\tableofcontents{{\singlespacing\thesis@tableofcontents}}
%    \end{macrocode}
% Set the table of contents entries to ``Chapter X'' and ``Appendix Y'' in the
% appropriate places and include leader dots.
%    \begin{macrocode}
\RequirePackage{titletoc}
\titlecontents{chapter}%
  [0pt]%
  {\vspace{1ex}}%
  {\bfseries\chaptername~\thecontentslabel\quad}%
  {\bfseries}%
  {~\mdseries\titlerule*[0.5em]{.}\bfseries\contentspage}
\g@addto@macro{\appendix}{%
\titlecontents{chapter}%
  [0pt]%
  {\vspace{1ex}}%
  {\bfseries\appendixname~\thecontentslabel\quad}%
  {\bfseries}%
  {~\mdseries\titlerule*[0.5em]{.}\bfseries\contentspage}
}
%    \end{macrocode}
% Set the table of contents depth to 0, corresponding to the Graduate College's
% preferred inclusion of chapters only.
%    \begin{macrocode}
\setcounter{tocdepth}{2}
%    \end{macrocode}
% \end{macro}
%
% \begin{macro}{\listoftables}
% \begin{macro}{\listoffigures}
% Similarly, redefine |\listoftables| and |\listoffigures| so
% that they use single spacing and add entries to the table of contents.
%    \begin{macrocode}
\let\thesis@listoftables=\listoftables
\def\listoftables{\newpage%
    \addcontentsline{toc}{chapter}{\listtablename}%
    {\singlespacing\thesis@listoftables}}
\let\thesis@listoffigures=\listoffigures
\def\listoffigures{\newpage%
    \addcontentsline{toc}{chapter}{\listfigurename}%
    {\singlespacing\thesis@listoffigures}}
%    \end{macrocode}
% \end{macro}
% \end{macro}
%
%
% \subsection{Other Frontmatter}
%
% \begin{environment}{abstract}
% Define the abstract environment.
%    \begin{macrocode}
\newenvironment{abstract}{\chapter*{\abstractname}}{}
%    \end{macrocode}
% \end{environment}
%
% \begin{environment}{dedication}
% The dedication environment just starts a new page and prints the dedication
% in the center in italics.
%    \begin{macrocode}
\newenvironment{dedication}{
    \newpage
    \leavevmode\vfill
    \begin{center}
    \itshape
    }{
    \end{center}
    \vskip 3ex
    \vfill
    \newpage
    }
%    \end{macrocode}
% \end{environment}
%
% \begin{environment}{acknowledgments}
% Define the acknowledgments environment.
%    \begin{macrocode}
\newenvironment{acknowledgments}{\chapter*{Acknowledgments}}{}
%    \end{macrocode}
% \end{environment}
%
% \begin{environment}{symbollist}
% \begin{environment}{abbrevlist}
% The \env{symbollist} environment can be used to create a list of symbols
% with the left column centered (for symbols).
% The \env{abbrevlist} environment has the left column left-justified
% (for abbreviations).
%    \begin{macrocode}
\newenvironment*{symbollist}[1][1in]{
    \begin{list}{}{\singlespacing
     \setlength{\leftmargin}{#1}
     \setlength{\labelwidth}{#1}
     \addtolength{\labelwidth}{-\labelsep}
     \setlength{\topsep}{0in}}%
     \def\makelabel##1{\hfil##1\hfil}%
    }{
    \end{list}}
\newenvironment*{abbrevlist}[1][1in]{
    \begin{symbollist}[#1]
    \def\makelabel##1{##1\hfil}}
    {\end{symbollist}}
%    \end{macrocode}
% \end{environment}
% \end{environment}
%
%
% \subsection{Page Numbering}
%
% Page numbers must be in one of three places, and must appear in the
% same place on \emph{every page}, including chapter openings.
% \begin{macro}{\ps@plain}
%    \begin{macrocode}
\RequirePackage{geometry}
\geometry{letterpaper}
\geometry{margin=1in}
\RequirePackage{fancyhdr}
% https://tex.stackexchange.com/a/33877
% Redefine plain style, which is used for titlepage and chapter beginnings
% From https://tex.stackexchange.com/a/30230/828
\fancypagestyle{plain}{%
    \renewcommand{\headrulewidth}{0pt}%
    \fancyhf{}%
    \fancyfoot[C]{\thepage}%
}
\pagestyle{plain}
%    \end{macrocode}
% \end{macro}
%
% \subsection{Body Formatting}
%
% To set up the actual formatting for the front matter of the thesis,
% simpliy specify roman page numbering.
%    \begin{macrocode}
\pagenumbering{roman}
%    \end{macrocode}
%
%    \begin{macrocode}
%</class|package>
%    \end{macrocode}
%\Finale